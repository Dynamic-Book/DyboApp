\documentclass{beamer}
\usepackage[utf8]{inputenc}
\usepackage{setspace}
\usepackage{hyperref}
\usepackage{amsmath}
\usepackage{pifont}
\usepackage{color}
% Just add % Just add % Just add \input{smalltalkEnv} to your file
% then you can use :
% \begin{lstlisting}[language=Smalltalk]
% false become: true.
% \end{lstlisting}

\usepackage{color}
\usepackage{listings}
\usepackage{etoolbox}
\usepackage{textcomp}

\definecolor{stComment}{rgb}{0.5,0.5,0.5}
\definecolor{stString}{rgb}{0.58,0,0.82}
\definecolor{stKeywords}{rgb}{0.21,0.55,0.7}
\definecolor{stNumbers}{rgb}{.5,0,0}

\newtoggle{InString}{}% Keep track of if we are within a string
\togglefalse{InString}% Assume not initally in string
\newcommand*{\ColorIfNotInString}[1]{\iftoggle{InString}{#1}{\color{stNumbers}#1}}%
\newcommand*{\ProcessQuote}[1]{#1\iftoggle{InString}{\global\togglefalse{InString}}{\global\toggletrue{InString}}}%

\lstdefinelanguage{Smalltalk}{
  keywordstyle=\color{stKeywords},
  commentstyle=\color{stComment},
  stringstyle=\color{stString},
  alsoletter=\#,
  identifierstyle=\idstyle, 
  showstringspaces=false,
  morekeywords={true,false,self,super,nil},
  sensitive=true, 
  morecomment=[s]{"}{"}, 
  morestring=[d]', 
  style=SmalltalkStyle,
  tabsize=2,
  basicstyle=\ttfamily,
  upquote=true,
}


\makeatletter%
\newcommand*\idstyle[1]{%
  \expandafter\id@style\the\lst@token{#1}\relax%
}
\def\id@style#1#2\relax{%
  \ifnum\pdfstrcmp{#1}{\#}=0%
  \ttfamily\color{stString} \the\lst@token%
  \else%
  \edef\tempa{\uccode`#1}%
  \edef\tempb{`#1}%
  \ifnum\tempa=\tempb%
  \ttfamily\color{blue} \the\lst@token%
  \else%
  \the\lst@token%
  \fi%
  \fi%
}

\lstdefinestyle{SmalltalkStyle}{ 
  literate=%
  {^}{{$\uparrow$}}1% 
  % {"}{{{\ProcessQuote{"}}}}1% Disable coloring within double quotes
  % {'}{{{\ProcessQuote{'}}}}1% Disable coloring within single quote
  {0}{{{\ColorIfNotInString{0}}}}1%
  {1}{{{\ColorIfNotInString{1}}}}1%
  {2}{{{\ColorIfNotInString{2}}}}1%
  {3}{{{\ColorIfNotInString{3}}}}1%
  {4}{{{\ColorIfNotInString{4}}}}1%
  {5}{{{\ColorIfNotInString{5}}}}1%
  {6}{{{\ColorIfNotInString{6}}}}1%
  {7}{{{\ColorIfNotInString{7}}}}1%
  {8}{{{\ColorIfNotInString{8}}}}1%
  {9}{{{\ColorIfNotInString{9}}}}1%
} 
 to your file
% then you can use :
% \begin{lstlisting}[language=Smalltalk]
% false become: true.
% \end{lstlisting}

\usepackage{color}
\usepackage{listings}
\usepackage{etoolbox}
\usepackage{textcomp}

\definecolor{stComment}{rgb}{0.5,0.5,0.5}
\definecolor{stString}{rgb}{0.58,0,0.82}
\definecolor{stKeywords}{rgb}{0.21,0.55,0.7}
\definecolor{stNumbers}{rgb}{.5,0,0}

\newtoggle{InString}{}% Keep track of if we are within a string
\togglefalse{InString}% Assume not initally in string
\newcommand*{\ColorIfNotInString}[1]{\iftoggle{InString}{#1}{\color{stNumbers}#1}}%
\newcommand*{\ProcessQuote}[1]{#1\iftoggle{InString}{\global\togglefalse{InString}}{\global\toggletrue{InString}}}%

\lstdefinelanguage{Smalltalk}{
  keywordstyle=\color{stKeywords},
  commentstyle=\color{stComment},
  stringstyle=\color{stString},
  alsoletter=\#,
  identifierstyle=\idstyle, 
  showstringspaces=false,
  morekeywords={true,false,self,super,nil},
  sensitive=true, 
  morecomment=[s]{"}{"}, 
  morestring=[d]', 
  style=SmalltalkStyle,
  tabsize=2,
  basicstyle=\ttfamily,
  upquote=true,
}


\makeatletter%
\newcommand*\idstyle[1]{%
  \expandafter\id@style\the\lst@token{#1}\relax%
}
\def\id@style#1#2\relax{%
  \ifnum\pdfstrcmp{#1}{\#}=0%
  \ttfamily\color{stString} \the\lst@token%
  \else%
  \edef\tempa{\uccode`#1}%
  \edef\tempb{`#1}%
  \ifnum\tempa=\tempb%
  \ttfamily\color{blue} \the\lst@token%
  \else%
  \the\lst@token%
  \fi%
  \fi%
}

\lstdefinestyle{SmalltalkStyle}{ 
  literate=%
  {^}{{$\uparrow$}}1% 
  % {"}{{{\ProcessQuote{"}}}}1% Disable coloring within double quotes
  % {'}{{{\ProcessQuote{'}}}}1% Disable coloring within single quote
  {0}{{{\ColorIfNotInString{0}}}}1%
  {1}{{{\ColorIfNotInString{1}}}}1%
  {2}{{{\ColorIfNotInString{2}}}}1%
  {3}{{{\ColorIfNotInString{3}}}}1%
  {4}{{{\ColorIfNotInString{4}}}}1%
  {5}{{{\ColorIfNotInString{5}}}}1%
  {6}{{{\ColorIfNotInString{6}}}}1%
  {7}{{{\ColorIfNotInString{7}}}}1%
  {8}{{{\ColorIfNotInString{8}}}}1%
  {9}{{{\ColorIfNotInString{9}}}}1%
} 
 to your file
% then you can use :
% \begin{lstlisting}[language=Smalltalk]
% false become: true.
% \end{lstlisting}

\usepackage{color}
\usepackage{listings}
\usepackage{etoolbox}
\usepackage{textcomp}

\definecolor{stComment}{rgb}{0.5,0.5,0.5}
\definecolor{stString}{rgb}{0.58,0,0.82}
\definecolor{stKeywords}{rgb}{0.21,0.55,0.7}
\definecolor{stNumbers}{rgb}{.5,0,0}

\newtoggle{InString}{}% Keep track of if we are within a string
\togglefalse{InString}% Assume not initally in string
\newcommand*{\ColorIfNotInString}[1]{\iftoggle{InString}{#1}{\color{stNumbers}#1}}%
\newcommand*{\ProcessQuote}[1]{#1\iftoggle{InString}{\global\togglefalse{InString}}{\global\toggletrue{InString}}}%

\lstdefinelanguage{Smalltalk}{
  keywordstyle=\color{stKeywords},
  commentstyle=\color{stComment},
  stringstyle=\color{stString},
  alsoletter=\#,
  identifierstyle=\idstyle, 
  showstringspaces=false,
  morekeywords={true,false,self,super,nil},
  sensitive=true, 
  morecomment=[s]{"}{"}, 
  morestring=[d]', 
  style=SmalltalkStyle,
  tabsize=2,
  basicstyle=\ttfamily,
  upquote=true,
}


\makeatletter%
\newcommand*\idstyle[1]{%
  \expandafter\id@style\the\lst@token{#1}\relax%
}
\def\id@style#1#2\relax{%
  \ifnum\pdfstrcmp{#1}{\#}=0%
  \ttfamily\color{stString} \the\lst@token%
  \else%
  \edef\tempa{\uccode`#1}%
  \edef\tempb{`#1}%
  \ifnum\tempa=\tempb%
  \ttfamily\color{blue} \the\lst@token%
  \else%
  \the\lst@token%
  \fi%
  \fi%
}

\lstdefinestyle{SmalltalkStyle}{ 
  literate=%
  {^}{{$\uparrow$}}1% 
  % {"}{{{\ProcessQuote{"}}}}1% Disable coloring within double quotes
  % {'}{{{\ProcessQuote{'}}}}1% Disable coloring within single quote
  {0}{{{\ColorIfNotInString{0}}}}1%
  {1}{{{\ColorIfNotInString{1}}}}1%
  {2}{{{\ColorIfNotInString{2}}}}1%
  {3}{{{\ColorIfNotInString{3}}}}1%
  {4}{{{\ColorIfNotInString{4}}}}1%
  {5}{{{\ColorIfNotInString{5}}}}1%
  {6}{{{\ColorIfNotInString{6}}}}1%
  {7}{{{\ColorIfNotInString{7}}}}1%
  {8}{{{\ColorIfNotInString{8}}}}1%
  {9}{{{\ColorIfNotInString{9}}}}1%
} 

\usetheme{AnnArbor}

% My preferences
\graphicspath{{images/}}
\newcommand{\tip}{\boldmath{\textcolor{red}{$\Rightarrow$}}}
\newcommand{\drgeo}{Dr.~Geo}
\newcommand{\cmark}{\text{\ding{51}}}
\newcommand{\xmark}{\text{\ding{55}}}

\title{Revisiting the Dynabook concept\\
\emph{for education}}

\author{Hilaire Fernandes}
\institute[DIP, Geneva]{Department of Public Instruction \\ Geneva}
\titlegraphic{
  \includegraphics[width=2cm]{ArmoirieGeneve.png}
}
\date{November 2023}

\begin{document}
\begin{frame}
  \titlepage
\end{frame}
%
\begin{frame}{About me}
  \fontsize{12pt}{30pt}\selectfont
  \begin{itemize}
  \item Educator in public school, Geneva, B.Math, Ma.Ed
  \item Computer scientist, Ma.CS, PhD.CS
  \item Free software enthusiast and user since 1998
  \item And of course, Smalltalk user since 2002
  \end{itemize}
\end{frame}
%
\begin{frame}{Contents}
  \begin{columns}
    \begin{column}{0.5\textwidth}
      \tableofcontents[hideallsubsections]      
    \end{column}
    \begin{column}{0.5\textwidth}
      \includegraphics[width=\textwidth]{image2-sm.png}      
    \end{column}
  \end{columns}
\end{frame}

\section{Why this presentation?}
\subsection{The cash register of education}
\begin{frame}
  \begin{itemize}
  \item The Dynabook in education is still mostly a concept
  \item In school, computerized environment used some time\\
    \tip\ compare to the other sectors of the society
  \end{itemize}

  \vspace{20pt}
  
  \tip\ Any serious Dynabook realization should be considered as a
  \emph{cash register of education}
\end{frame}
%
\begin{frame}{Not this cash register}
  \begin{center}
      \includegraphics[width=0.4\textwidth]{Old_National_Cash_Register.jpg}
    \end{center}
\end{frame}
%
\begin{frame}{But more like this one}
  \begin{center}
    \includegraphics[width=0.5\textwidth]{Telpo-C68-02.png}
  \end{center}
  
  \tip\ What about a dedicated hardware and software environment for a
  meaningful use in education
\end{frame}
\subsection{To get you involved}
\begin{frame}
  \begin{itemize}
  \item Educator, professor
  \item Benefactor
  \item Student at university
  \item Econimist
  \item Software developer
  \item Project manager
  \item Hardware specialist
  \item Designer
  \item System administrator
  \item ...
  \end{itemize}
\end{frame}

\section{In essence, what is Dynabook?}
\begin{frame}
  \fontsize{14pt}{8pt}\selectfont
\begin{center}
  A vehicle for dynamic models of knowledge\\ the user can design and/or operate on
\end{center}
\end{frame}
%
\begin{frame}{Dynamic model of the \emph{Newton Telescop}}
  The learner can operate a different level of knowledge light beam,
  focus point (change the mirror curve)
\begin{center}
  \includegraphics[width=0.3\textwidth]{Newton.png}
\end{center}
\end{frame}

\section{Changing the Point of View}
\begin{frame}{Teacher}
  What is a teacher?
  \vspace*{10pt}
  \begin{itemize}
  \item Yes, an educator. Manipulating, designing, sharing knowledge.\\
    \tip\ Library of dynamic knowledge models, scriptable with a DSL
    and/or GUI (think \textbf{\drgeo}, \textbf{keynote Tuesday 9/11 at 11:00})

  \item ... but also a manager
    \begin{itemize}
    \item managing student
    \item managing assignment
    \item managing grade
    \item managing meeting
    \item managing parents
    \item ...
    \end{itemize}
  \end{itemize}
  \vspace*{10pt}

  Any realistic Dynabook revisit should take theses aspects in
  consideration.
\end{frame}
%
\begin{frame}{Kid}
  Can we reduce the bag weight?

  \vspace*{10pt}

  I have seen lightweight students assigned with mountain of materials:
  \begin{itemize}
  \item $>$10 binders
  \item $>$10 books
  \item $>$10 activity files
  \item numerous notebooks
  \end{itemize}

  \vspace*{10pt}
  
  Any realistic Dynabook revisit should take theses aspects in
  consideration.

\end{frame}
%
\begin{frame}{Software environment}
  What do we need?

  \vspace{10pt}
  
  \begin{itemize}
  \item Free software from the basement (OS) to the attic (end user
    applications)
  \item Rapid prototyping
  \item A malleable environment to develop knowledge models with state
    of the art visual representation
  \item Easy to implement DSL to script knowledge models
  \item Portable to different hardware architecture
  \end{itemize}

  \vspace*{10pt}

  \tip\ Cuis-Smalltalk\footnote[frame]{assumed biais} to develop end user applications and knowledge
  models
\end{frame}
%
\begin{frame}{Free Hardware}
  \fontsize{14pt}{8pt}\selectfont
  \begin{center}
    Design there \& there\\
    \vspace*{10pt}
    Manufacture anywhere
  \end{center}
\end{frame}

\begin{frame}{Economic}
  Large scale adoption in one place, also require local economic
  benefits on that place:
  \begin{itemize}
  \item Software support
  \item Assembling/Manufacturing
  \item Repairing
  \item Training
  \end{itemize}

  \vspace*{10pt}
  
  \tip\ Free software \& hardware as prerequisites
  
\end{frame}

\section{Where are we?}
\begin{frame}{Is there any plan?}
\begin{center}
  \includegraphics[width=0.8\textwidth]{anyPilot.jpeg}
\end{center}
\end{frame}
\begin{frame}{Roughly}
  Iterations
  \begin{enumerate}
  \item \textbf{Develop the Dynabook app} (me, but join!)
  \item Test Dynabook app in school and iterate with the development (1 or 2 teachers)
  \item Develop hardware prototype with existing hardware
  \item Develop Dynabook operating system
  \item Test Dynabook app in school and iterate with the development
    (tenth of teachers)
  \item Test Dynabook in school and iterate on the hardware and
    software (1 or 2 teachers/students)
  \item Test Dynabook hardware and software with one classroom (30 users, students and teachers)
  \end{enumerate}
\end{frame}

\begin{frame}{Visual Concepts}
\begin{center}
  \includegraphics[width=0.8\textwidth]{image6-sm.png}
\end{center}
\end{frame}
%
\begin{frame}{Management - Concept}
\begin{center}
  \includegraphics[width=0.9\textwidth]{image17.png}
\end{center}
\end{frame}
%
\begin{frame}{Management Viewer}
\begin{center}
  \includegraphics[width=0.5\textwidth]{prefViewer.png}
\end{center}
\end{frame}
%
\begin{frame}{Management Editor}
\begin{center}
  \includegraphics[width=0.4\textwidth]{prefEditor.png}
\end{center}
\end{frame}
%
\begin{frame}{Knowledge environment}
\begin{center}
  \includegraphics[width=0.8\textwidth]{image3.png}
\end{center}
\end{frame}
%
\begin{frame}{Paper Morph}
\begin{center}
  \includegraphics[width=0.5\textwidth]{image8.png}
\end{center}
\end{frame}
%
\begin{frame}{Paper Morph -- Pressure emulation}
\begin{center}
  \includegraphics[width=0.8\textwidth]{image9.png}
\end{center}
\end{frame}
\section{How to get involved?}
\begin{frame}{1. Who \& What?}
  \begin{itemize}
  \item Educators

    \tip\ Write the specifications of interactive knowledge models in
    your domain

    \tip\ Review, compile existing pedagogical resources -- under free
    license
  \vspace*{10pt}

  \item Software developers

    \tip\ Participate to the Dynabook.app design
    
    \tip\ Code with Cuis-Smalltalk interactive knowledge model and DSL
  \vspace*{10pt}
    
  \item Professors

    \tip\ Student projects
  \end{itemize}
\end{frame}
%
\begin{frame}{2. Who \& What?}
  \begin{itemize}
  \item Economists

    \tip\ Prospective on economic benefits

    \tip\ Environmental impacts
  \vspace*{10pt}

  \item Hardware specialists

    \tip\ Participate to the Dynabook hardware specification and
    design
  \vspace*{10pt}

  \item Benefactors

    \tip\ Set up a foundation to support the software and hardware
    specifications, design and development
    
  \end{itemize}
\end{frame}
%
\begin{frame}
  \begin{quote}
    If everything you try works, you aren't trying hard enough.
      \end{quote}
  \begin{flushright}
    -- Gordon Moore
  \end{flushright}
  \begin{center}
    \url{http://github.com/hilaire/dynabook}
  \end{center}
\end{frame}

\end{document}